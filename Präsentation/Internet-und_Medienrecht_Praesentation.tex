\documentclass{beamer}

\usepackage[ngerman]{babel}
\usepackage[utf8]{inputenc}
\usepackage{hyperref}

\mode<presentation>{
	\definecolor{LogoRed}{RGB}{219,00,31}
	\definecolor{LogoGray}{RGB}{86,86,80}
	
	\useoutertheme[width=2.1cm]{sidebar}
	\useinnertheme{rounded}
	\setbeamercolor{normal text}{fg=black,bg=white}
	\setbeamercolor{palette sidebar primary}{use=normal text,fg=normal text.fg}
	\setbeamercolor{title}{fg=LogoRed}
	\setbeamercolor{frametitle}{fg=LogoRed}
	\setbeamercolor{structure}{fg=LogoRed}
	\setbeamercolor{section in sidebar}{fg=LogoRed}
	\setbeamercolor{subsection in sidebar}{fg=LogoRed}
	
	\usefoottemplate{\vbox{\tinycolouredline{LogoGray!25}{\hspace{4pt}\hspace{15pt}\insertdate\hfill\insertshortinstitute\hfill \insertframenumber{}/\inserttotalframenumber}}}
	
	\setbeamertemplate{section in toc}{\inserttocsectionnumber.~\inserttocsection\par}
	\setbeamertemplate{subsection in toc}{\hspace*{2em}\inserttocsectionnumber.\inserttocsubsectionnumber~\inserttocsubsection}
	\AtBeginSubsection[] {
		\begin{frame}<beamer>
			\frametitle{Outline}
			\tableofcontents[currentsection,sectionstyle=show/show,subsectionstyle=show/shaded/hide]
		\end{frame}
	}
}

\title[Internet- und Medienrecht]{Rechte an einer Website}
\author[Lukas Dittmann \\und\\ Diana Irmscher]{Lukas Dittmann \\\texttt{dittmann@hm.edu}\\ Diana Irmscher\\\texttt{irmscher@fs.cs.hm.edu}}
\institute[Hochschule für angewandte Wissenschaften München]{Fakultät für Informatik und Mathematik}
\logo{\includegraphics[height=1cm]{logo_hm.png}}
\date{\today}

\setbeamertemplate{navigation symbols}{}

\begin{document}
	\maketitle
	
	\section{Begrüßung}
	
	\begin{frame}[t]
		\frametitle{Vorstellung}
		\begin{columns}
			\column{.47\textwidth}
			
			\textbf{Lukas Dittmann}
			\\
			Informatik, Bachelor
			\column{.47\textwidth}
			\textbf{Diana Irmscher}
			\\
			Informatik, Bachelor
		\end{columns}
	\end{frame}
	
	\section{Agenda}
	
	\begin{frame} %% Agenda zeigen, was noch ansteht
		\frametitle{Agenda}
		\tableofcontents
	\end{frame}
	
	\section{}
	
	\begin{frame}
		\frametitle{Über was reden wir heute?}
		\textbf{Thema: Rechte an einer Website}
		\\
		\begin{itemize}
			\item Überblick über die mit einer Website verbundenen Rechte:
			\begin{itemize}
				\item Welche Rechtsqualität besitzt eine Domain?
				\item Welche Rechte kann der Inhaber einer Domain daraus ableiten?
				\item Welche Stellung hat die DENIC?
				\item Für den Fall einer angenommenen Rechtsverletzung: Auf welcher rechtlichen Grundlage kann der Inhaber einer Marke oder eines Namens gegen einen Domaininhaber vorgehen? Welches sind die Voraussetzungen?
			\end{itemize}
		\end{itemize}
	\end{frame}
	
	\section{Begriffdefinition}
	
	\begin{frame}
		\frametitle{Definition fachbezogener Begriffe}
		%%hier noch die Begriffe einfügen
	\end{frame}
	
	\section{Rechtsqualität einer Domain}
	
	\begin{frame}
		\frametitle{Welche Rechte gelten für eine Domain}
		\begin{itemize}
			\item für das Domainrecht gibt es keine einheitliche Rechtssprechung und Gesetzgebung
			\item vielmehr gelten folgende Gesetzte
			\begin{itemize}
				\item § 12 BGB
				\item § 5 MarkenG
				\item usw.
			\end{itemize}
		\end{itemize}
	\end{frame}
	
	\section{Rechte einer Domain ableiten}
	
	\begin{frame}
		\frametitle{Welche Rechte kann der Inhaber einer Domain daraus ableiten?}
		\begin{itemize}
			\item Domaininhaber ist Vertragspartner mit DENIC
			\item ist materiell Berechtigter
		\end{itemize}
	\end{frame}
	
	\section{Domain: Erklärung und Stellung}

	\begin{frame}
		\frametitle{Was ist überhaupt eine Domain?}
		Eine Domain hat mehrere Aufgaben in der Internetwelt.
		\\
		\bigskip
		In diesem Vortrag geht es hauptsächlich darum, dass Domains eine für den Menschen lesbare und einprägsame Zeichenkombination bietet, die der Computer anschließend in tatsächliche Adressen umsetzt.
	\end{frame}
	
	\begin{frame}
		\frametitle{Was ist überhaupt eine Domain?}
		Ein Beispiel:
		\\
		Der Webserver der Hochschule München hat die IP-Adresse 129.187.244.229. Diese ist für den Menschen auch unter \glqq hm.edu\grqq erreichbar.
	\end{frame}
	
	
	\begin{frame}
		\frametitle{Was ist überhaupt eine Domain?}
		Wie setzt sich eine Domain zusammen?
		\\
		
		Eine Domain ist in mehrere Bausteine (entspricht auch grob den Zuständigkeiten) unterteilt und wird hierarchisch von rechts nach links gegliedert. Die einzelnen Bereiche sind durch Punkte getrennt.
	\end{frame}
	
	\begin{frame}
		\frametitle{Wie setzt sich eine Domain zusammen?}
		Im vorherigen Beispiel wäre also die Unterteilung \glqq.edu\grqq und \glqq hm\grqq. \glqq.edu\grqq steht am höchsten Punkt der Hierarchie und wird deshalb auch Top-Level-Domain bezeichnet (oder kurz TLD)
	\end{frame}
	
	
	\begin{frame}
		\frametitle{Wie setzt sich eine Domain zusammen?}
		Der drunter liegende Baustein wäre \glqq hm\grqq. Dieser wird folglich dann Second-Level Domain genannt. Diese Second-Level-Domains werden von den beauftragten Stellen der TLD-Verwaltung vergeben.
	\end{frame}
	
	\section{Welche Stellung hat die Denic?}
	
	\begin{frame}
		\frametitle{Wer ist für \glqq.de\grqq zuständig?}
		\glqq.de\grqq ist eine sogenannte Geo-TLD und dessen Second-Level-Domains werden an Websiten vergeben, die in Deutschland tätig sind oder Diente für Deutschland anbieten. Sie werden von der DENIC vergeben.
	\end{frame}
	
	\begin{frame}
		\frametitle{Welche Stellung hat die Denic?}
		Wie bereits vorhin beschrieben, ist die DENIC die Vergabestelle für die Geo-TLD, die Deutschland zugeordnet ist. Vergabestelle, nicht mehr und nicht weniger.
	\end{frame}
	
	
	\begin{frame}
		\frametitle{Welche Stellung hat die Denic?}
		Immer wieder wird die DENIC in Rechtsstreitigkeiten entweder direkt oder indirekt vor Gericht gezogen. Da es keine gesetzliche Grundlage gibt werden Gerichtsurteile als Basis genutzt. Hier ein paar Beispiele: 
	\end{frame}
	
	\begin{frame}
		\frametitle{Welche Stellung hat die Denic?}
		Der Fall \glqq Lufthansa gegen DENIC\grqq\\
		\bigskip
		Aktenzeichen 2-06 O 706/08 Landgericht Frankfurt
		\begin{itemize}
			\item Verbot der Herausgabe von ähnlich klingenden(lufthanza.de) und ergänzenden(lufthansa-jobs.de) Domains ohne vorherige Prüfung
			\item Klage wurde abgewiesen
		\end{itemize}
	\end{frame}
	
	
	\begin{frame}
		\frametitle{Welche Stellung hat die Denic?}
		Der Fall \glqq Lufthansa gegen DENIC\grqq
		Die Begründung:
		\begin{itemize}
			\item Rechtsverletzungen nur gegen Störer
			\item Prüfungspflicht wurde nicht vernachlässigt, da nicht anwendbar.\\(Vgl Urteil zu "'ambiente.de"')
			\item "'Negativliste"' reicht auch nicht aus, da auch gültige Domains gesperrt werden.\\ z.B.: "'freilufthansaparksommerfestival.de"'
			\item Verstoß gegen das "'First-Come-First-Served"'-Prinzip
		\end{itemize}
	\end{frame}

	\begin{frame}
		\frametitle{Welche Stellung hat die Denic?}
		Der Fall \glqq Lufthansa gegen DENIC\grqq\\
		Die Begründung:
		\begin{itemize}
			\item Es kann nicht erwartet werden, dass DENIC-Mitarbeiter Experten in Markenrecht sind\\(Vgl Urteil zu "'ambiente.de"' oder in der Entscheidung zu "'viagratip.de"')
			\item Die DENIC bietet die Möglichkeit, den verantwortlichen Störer ab zu fragen (öffentliche WHOIS-Datenbank)
		\end{itemize}
	\end{frame}
	
	\begin{frame}
		\frametitle{Welche Stellung hat die Denic?}
		BGH-Urteil \glqq Messe Frankfurt Ambiente gegen DENIC\grqq\\
		\textbf{NACHTRAGEN}
		\begin{itemize}
			\item Herausgabe der Domain ambiente.de
			\item DENIC hat die Adresse nur gesperrt, aber nicht dem alten Besitzer entrissen (DISPUTE-Eintrag)
		\end{itemize}
	\end{frame}
	
	
	\begin{frame}
		\frametitle{Einschub: Was ist ein DISPUTE-Eintrag?}
		\begin{itemize}
			\item Während der automatischen Erstregistrierung von "'.de"'-Domains wird keine Überprüfung auf Rechtsverletzungen vorgenommen
			\item Sollte eine solche Verletzung aufgetreten sein, muss sich der Geschädigte direkt an den Störer wenden
			\item Der DISPUTE Eintrag ist Möglichkeit die Domain zu sperren, damit diese nicht verkauft oder während dem Rechtsstreit neu vergeben wird
			\item Nach Rückgabe der Domain durch den alten Besitzer, hat der Antragssteller die Option diese neu zu erwerben
			\item Auch hier gilt das "'First-Come-First-Served"'-Prinzip
		\end{itemize}	
	\end{frame}	
	
	\begin{frame}
		\frametitle{Welche Stellung hat die Denic?}
		BGH-Urteil \glqq Messe Frankfurt Ambiente gegen DENIC\grqq\\
		Das Urteil:
		\begin{itemize}
			\item Die DENIC kann nur belastet werden, wenn sie
			\begin{itemize}
				\item den Rechtsverstoß fördere
				\item keine Handlung bei klarem Verstoß einleite
			\end{itemize}
			\item Das BGH-Urteil vergleicht die Haftung mit der Haftung bei der Annahme von Werbung in Pressemedien\\(Diese sind auch nicht für eine Markenrechtsverletzung verantwortlich)
		\end{itemize}
	\end{frame}
	
	\begin{frame}
		\frametitle{Welche Stellung hat die Denic?}
		BGH-Urteil \glqq Messe Frankfurt Ambiente gegen DENIC\grqq\\
		Das Urteil:
		\begin{itemize}
			\item Festlegung der Prüfungspflicht der DENIC:\\Tritt nur in Kraft
			\begin{itemize}
				\item nach der erfolgreichen Erstregistrierung
				\item wenn der beeinträchtigte Dritte den Verstoß anzeigt
				\item der Verstoß ohne tiefere Kenntnisse im Markenrecht erkennbar ist
			\end{itemize}
		\end{itemize}
	\end{frame}
	
	\section{Möglichkeiten bei einer Rechtsverletzung}
	
	\begin{frame}
		\frametitle{Möglichkeiten bei einer Rechtsverletzung}
		\begin{itemize}
			\item Unterlassung\\mögliche Paragraphen:
			\begin{itemize}
				\item §14 Abs. 5 MarkenG\\Missachtung der Bewerbung und Nutzung der geschützten Zeichen einer Marke
				\item §15 Abs. 4 MarkenG\\Verletzung der exklusiven Rechte an einem geschützten Namens
				\item §3 und §4 UWG\\Irreführung eines Verbrauchers durch Nutzung eingetragener Marken und dessen Konsequenz
				\item §12 i.V.m. § 1004 BGB\\Beschreibt ähnliche Möglichkeiten der Unterlassung, aber für den Zivilen Gebrauch
			\end{itemize}	
		\end{itemize}
	\end{frame}
	
	
	\begin{frame}
		\frametitle{Möglichkeiten bei einer Rechtsverletzung}
		\begin{itemize}
			\item Schadeneratz\\mögliche Paragraphen:
			\begin{itemize}
				\item §14 Abs. 6 MarkenG\\Missachtung der Bewerbung und Nutzung der geschützten Zeichen einer Marke
				\item §15 Abs. 5 MarkenG\\Verletzung der exklusiven Rechte an einem geschützten Namens
				\item §823 Abs 2 BGB\\Rechtliche Grundlage für Schadenersatz bei einer Verletzung von Rechten im zivilen Bereich
				\item §826 BGB\\Grundlage für Schadenersatz bei sittenwidriger Absicht
				\item §9 UWG\\Paragraph für den Schadenersatz von Mitbewerber gegenüber Mitbewerbern
			\end{itemize}
		\end{itemize}
	\end{frame}
	
	\section{Ausblick}
	
	\begin{frame}
		\frametitle{Wie sieht es bei .com und ähnlichen aus?}
		"'.com"', "'.org"' und "'.info"' sind so genannte Generic Doamins, die unter bestimmten Auflagen an Kunden weltweit verteilt werden.\\
		Das Problem:
		\begin{itemize}
			\item Sie werden weltweit vergeben
			\item "'First-Come-First-Served"'-Prinzip
			\item Jeder Besitzer beruft sich auf seine länderspezifischen Gesetze
			\item Einfachste Lösung: außergerichtliche Einigung
		\end{itemize}
	\end{frame}
	
	\begin{frame}
		\frametitle{Fazit}
		\begin{itemize}
			\item in den letzten Jahren immer weniger Urteile zu Domainrecht
			\item mehr Verfahren zu Admin-C-Haftung
			\item oft außergerichtliche Einigungen
		\end{itemize}
	\end{frame}
	
	\begin{frame}
		\frametitle{}
		\center
		\textbf{Fragen?}
	\end{frame}
	
\end{document}